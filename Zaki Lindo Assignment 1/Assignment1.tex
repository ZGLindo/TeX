\documentclass[11pt]{amsart}

\begin{document}


 {\bf  Assignment 1 - Fall 2017 Zaki G. Lindo}
 
 \vskip .2in

\noindent Assignment 1: Due Monday September 4, 2017, no later than 8pm.\\
$p. 12-14$ \# 16, 42, 47\\
$p. 36-37$ \# 7, 13\\


\vskip .1in

$p. 12-14$\\

\#16. For the statement about nonempty sets A, B, C, either prove that the statement is true or find a counterexample with nonempty sets that make it fail.\begin{center}
If $A \subseteq (B \cap C)$ then $A \subseteq B$ and $A \subseteq C$.\\
\end{center}
\vskip .2in

\textbf{Proof:} Let $A$, $B$, and $C$, be nonempty sets. Let $A \subseteq(B \cap C)$. We must show that $A \subseteq B$ and $A \subseteq C$. As $A \subseteq(B \cap C)$, for all $x \in A$, $x \in (B \cap C)$. By definition of $intersects$, for all $x \in B \cap C$, $x \in B$ and $x \in C$. As we showed $x \in B$, $x \in C$, and $x$ was arbitrary, $A \subseteq B$ and $A \subseteq C$. Thus, If $A \subseteq (B \cap C)$ then $A \subseteq B$ and $A \subseteq C$.
\vskip 0.4in

\#42. Determine if the function below is injective, surjective, or bijective. Prove it is true or give a counterexample for each property.\\
\begin{center}
$h:\mathbb{Z} \to \mathbb{Z}$ where $h(x) = 2x-5$.\\
\end{center}
\
\vskip 0.1in
(Injective)
\textbf{Proof by Contradiction:} Let $h:\mathbb{Z}\to\mathbb{Z}$ where $h(x)=2x-5.$ For a contradiction, assume $h$ is not injective. That is, for all $m,n \in \mathbb{Z}, m\neq n$ and $h(m)=h(n)$. As $h(m)=h(n), 2m-5=2n-5.$ It follows that $2m=2n$. Thus, $m=n$. As $m\neq n$ and $m=n$, we have a contradiction. Therefore, $h$ is injective.
\vskip 0.2in
(Surjective)
\textbf{Counter Example}: Let $h(x)=2$. Solve for $x.$
\\
\\$2=2x-5$
\\$7=2x$
\\$7/2=x.$
\\
\\As $7/2\notin\mathbb{Z}$ There exists $y\in\mathbb{Z}$ such that for all $x\in\mathbb{Z}, h(x)\neq y.$ Therefore, $h$ is not surjective.
\vskip 0.2in
(Bijective) As $h$ is not surjective, $h$ is not bijective.
\vskip 0.4in

\#47. Carefully Prove the statement below.\\
\begin{center}
If $f$ and $g$ are functions with $f:A \to B$, $g:B \to C$ and \\  $g \circ f$ is surjective then $g$ is surjective.\\
\end{center}
\vskip 0.2in
\textbf{Proof:} Suppose $f$ and $g$ are functions with $f:A\to B$, $g:B\to C$ and $g\circ f$ is surjective. We want to show that $g$ is surjective. We know $g\circ f(x)=g(f(x)).$ Let $x\in A$. As $g(f(x))$ is surjective, there exists $y\in C$ such that $g(f(x))=y.$ Let $z\in B$ such that $f(x)=z$. Now we have that there exists $y\in C$ such that $g(z)=y$. As $x$ was arbitrary, we can say that $z$ was arbitrary. As $z$ was arbitrary, for all $z\in B,$ there exists $y\in C$ such that $g(z)=y$. Thus, g is surjective.


\vskip 0.4in

$p. 36-37$\\

\#7. Determine if the rule $x \ast y = xy$ is an operation on the set of negative integers.\\
\vskip 0.2in

This is not an operation on the set of negative integer. There exists $x,y \in \mathbb{-Z}$ such that $x\ast y \notin \mathbb{-Z}:$
\\

\textbf{Counter Example}: Let $x=-1$. Let $y=-2$. This gives us $x\ast y= (-1)(-2)=2\notin\mathbb{-Z}.$ As $2\notin\mathbb{-Z}, x\ast y$ is not an operation on the set of negative integers.

\vskip 0.4in
\#13. Determine if the operation defined below on the set $\mathbb{Z}$ is associative, commutative, has an identity, and if each element has an inverse. Either prove or give a counterexample for each property.\\
\begin{center}
For $a,b \in \mathbb{Z}$, $a \ast b = a+2b - 1$.\\
\end{center}
\vskip 0.2in
(Commutative): For all $a,b\in \mathbb{Z}$, $a\ast b=b\ast a.$
\\
\\Let $a=2.$ Let $b=4$. 
\\$a\ast b=2+2(4)-1=9.$ 
\\$b\ast a=4+2(2)-1=7.$
\\$9\neq 7$. 
\\Thus, $\ast$ is not commutative on the set $\mathbb{Z}$.
\\
\\
\\(Associative): For all $a,b\in \mathbb{Z}$, $(a\ast b)\ast c=a\ast (b\ast c).$
\\
\\Let $a=1.$ Let $b=2.$ Let $c=-3.$
\\$(a\ast b)\ast c=(1+2(2)-1)+(2)(-3)-1=4-6-1=-3.$
\\$a\ast (b\ast c)=1+2(2+2(-3)-1)-1)-1=1+2(-5)-1=-10$
\\$-10\neq -3$. 
\\Thus, $\ast$ is not associative on the set $\mathbb{Z}.$
\\
\\
\\(Identity): There exists $e\in \mathbb{Z}$ such that for all $a\in \mathbb{Z}, e\ast a=a$ and $a\ast e=a.$
\\
\\Let $a = 1$.Let $e$ be the identity for $1$. Then $1\ast e=1.$
\\$1\ast e= 1+2e-1=1$,
\\$2e=1$,
\\$e=1/2.$ But $e\notin\mathbb{Z}.$ Therefore, $1$ does not have an identity $e \in\mathbb{Z}$.
\\
\\
\\(Inverse):There exists $i\in \mathbb{Z}$ such that for all $a\in\mathbb{Z}, a\ast i=e$.
\\
\\If we observe the counter example we used in our (Identity) presentation we will see $\ast$ does not have an inverse for that counter example, because there isn't an identity element $e\in\mathbb{Z}$ that we can say $a\ast i=e$.
\\
\\Due to this, we can say not every element has an inverse under
 $\ast$.
\end{document}