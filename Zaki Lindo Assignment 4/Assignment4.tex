\documentclass[11pt]{amsart}

\begin{document}


 {\bf Zaki Glenroy Lindo, Assignment 4}
 
 \vskip .2in

\noindent Assignment 4: Due Monday October 2, 2017, no later than 8pm.\\

$p. 38-39$ \# 42, 43, 46, 60 , 66\\


\vskip .1in

\#42. Prove or disprove that $H = \{0, 2, 4, 6, 16\}$  is a subgroup of $(\mathbb{Z}_{18}, +_{18})$\\
\\
Counter Example:\\
\\Let $x=2.$ 
\\Let $y=6$.
\\
\\As we defined $H$, $x,y\in H$. However, $x\ast y=2+_{18}6=8$. $8\notin H$, therefore, H is not closed under its operation, $+_{18}$. Thus, $H$ is not a subgroup of $(\mathbb{Z}_{18}, +_{18})$.
\\
\vskip 0.2in


\#43. Prove or disprove that $H = \{(0,0), (0,2), (1, 0), (1,2)\}$  is a subgroup of $\mathbb{Z}_{2} \times \mathbb{Z}_{4}$under the operation using $+_2$ in the first coordinate and $+_4$ in the second coordinate.\\
\\
\\
\begin{tabular}{c | c  c  c  c }
	$+_2 \times +_4$ & (0,0) & (0,2) & (1,0) & (1,2)\\
	\hline
	(0,0) & (0,0) & (0,2) & (1,0) & (1,2) \\
	
	(0,2) & (0,2) & (0,0) & (1,2) & (1,0) \\
	
	(1,0) & (1,0) & (1,2) & (0,0) & (0,2) \\
	
	(1,2) & (1,2) & (1,0) & (0,2) & (0,0)\\
\end{tabular}

\vskip 0.2in
Proof: We want to show that H is a subgroup of $\mathbb{Z}_{2} \times \mathbb{Z}_{4}$ under the operation using $+_2$ in the first coordinate and $+_4$ in the second coordinate. We need to show $H\neq \emptyset ^1$, H is closed under the operation$^2$, and H is closed under inverses$^3$:
\\1) ($H\neq \emptyset$): As $H = \{(0,0), (0,2), (1, 0), (1,2)\}$, there are elements in $H$. Thus, $H\neq \emptyset$.
\\
\\2) (H is closed under the operation): We want to show that for all $x,y\in H$, $x\ast y\in H$. Looking at the Cayley Table, we can see that for all $x,y\in H$, $x\ast y\in H$. Thus, H is closed under the operation.
\\
\\3) (H is closed under inverses): We want to show that for all $a\in H$, $a^{-1}\in H$. \\
\\a=(0,0): \\Looking at the Cayley Table, we see that (0,0) $\ast$ (0,0) = (0,0). As (0,0) is the identity of H(as seen on the Cayley Table), (0,0) is the inverse of (0,0). Thus, $(0,0)^{-1}\in H$.\\
\\a=(0,2): \\Looking at the Cayley Table, we see that (0,2) $\ast$ (0,2) = (0,0). As (0,0) is the identity of H(as seen on the Cayley Table), (0,2) is the inverse of (0,2). Thus, $(0,2)^{-1}\in H$.\\
\\a=(1,0): \\Looking at the Cayley Table, we see that (1,0) $\ast$ (1,0) = (0,0). As (0,0) is the identity of H(as seen on the Cayley Table), (1,0) is the inverse of (1,0). Thus, $(1,0)^{-1}\in H$.\\
\\a=(1,2): \\Looking at the Cayley Table, we see that (1,2) $\ast$ (1,2) = (0,0). As (0,0) is the identity of H(as seen on the Cayley Table), (1,2) is the inverse of (1,2). Thus, $(1,2)^{-1}\in H$.\\
\\
\\As for each element $a\in H$, $a^{-1}\in H$, H is closed under inverses.
\\
\\Hence, as $H\neq \emptyset ^1$, H is closed under the operation$^2$, and H is closed under inverses$^3$, H is a subgroup of $\mathbb{Z}_{2} \times \mathbb{Z}_{4}$ under the operation using $+_2$ in the first coordinate and $+_4$ in the second coordinate.
\\
\\

\vskip 0.2in


\#46. Determine if the set H = $\left\{{\frac{1}{n} : n \in \mathbb{Z}, n\neq 0}\right\}$ is a subgroup of $(\mathbb{Q}^{\ast}, \cdot)$ (nonzero rational numbers under multiplication). Either prove that it is or give a specific example of how it fails.\\
\\Counter Example:
\\
\\Let $n=12$. We know that for 12, $\frac{1}{12}\in H$. 
\\Under the usual multiplication on nonzero rational numbers, $(\frac{1}{12})^{-1}=12$. However, $12\notin H$. Thus, H is not closed under inverses.
\\
\\As H is not closed under inverses, H is not a subgroup of ($\mathbb{Q^*},\cdot$) 
\\
\vskip 0.2in


\#60. Determine if the function $f:M_2(\mathbb{R}) \to \mathbb{R}$ defined by $f\left({\left[{ \begin{tabular}{c c}  a & b\\ c & d\end{tabular}}\right]}\right) = a+b$ is a homomorphism. Note that $M_2(\mathbb{R})$ is a group under matrix addition and $\mathbb{R}$ is a group under usual real number addition.\\
\\Proof: We want to show that $f:M_2(\mathbb{R}) \to \mathbb{R}$ defined by $f\left({\left[{ \begin{tabular}{c c}  a & b\\ c & d\end{tabular}}\right]}\right) = a+b$ is a homomorphism. That is, for every $x,y\in M_2(\mathbb{R})$, $f(x+y)=f(x)+f(y)$. Let $x={\left[{ \begin{tabular}{c c}  a & b\\ c & d\end{tabular}}\right]}$ and let $y={\left[{ \begin{tabular}{c c}  e & f\\ g & h\end{tabular}}\right]}$. \\
\\We want to show that $f\left({\left[{ \begin{tabular}{c c}  a & b\\ c & d\end{tabular}}\right]}+{\left[{ \begin{tabular}{c c}  e & f\\ g & h\end{tabular}}\right]}\right)=f\left({\left[{ \begin{tabular}{c c}  a & b\\ c & d\end{tabular}}\right]}\right)+f\left({\left[{ \begin{tabular}{c c}  e & f\\ g & h\end{tabular}}\right]}\right)$. \\
\\Now, $f\left({\left[{ \begin{tabular}{c c}  a & b\\ c & d\end{tabular}}\right]}+{\left[{ \begin{tabular}{c c}  e & f\\ g & h\end{tabular}}\right]}\right)=f\left({\left[{ \begin{tabular}{c c}  a+e & b+f\\ c+g & d+h\end{tabular}}\right]}\right)$ by Matrix multiplication. By f, $f\left({\left[{ \begin{tabular}{c c}  a+e & b+f\\ c+g & d+h\end{tabular}}\right]}\right)=(a+e)+(b+f)$. By associativity and commutativity of addition of real numbers, $(a+e)+(b+f)= a+b+e+f$. Thus, $f\left({\left[{ \begin{tabular}{c c}  a & b\\ c & d\end{tabular}}\right]}+{\left[{ \begin{tabular}{c c}  e & f\\ g & h\end{tabular}}\right]}\right)=a+b+e+f$\\
\\Now, $f\left({\left[{ \begin{tabular}{c c}  a & b\\ c & d\end{tabular}}\right]}\right)+f\left({\left[{ \begin{tabular}{c c}  e & f\\ g & h\end{tabular}}\right]}\right)=(a+b)+(e+f)$. By associativity of addition of real numbers, $(a+b)+(e+f)=a+b+e+f$. Thus, $f\left({\left[{ \begin{tabular}{c c}  a & b\\ c & d\end{tabular}}\right]}\right)+f\left({\left[{ \begin{tabular}{c c}  e & f\\ g & h\end{tabular}}\right]}\right)=a+b+e+f$\\
\\Therefore, $f\left({\left[{ \begin{tabular}{c c}  a & b\\ c & d\end{tabular}}\right]}+{\left[{ \begin{tabular}{c c}  e & f\\ g & h\end{tabular}}\right]}\right)=f\left({\left[{ \begin{tabular}{c c}  a & b\\ c & d\end{tabular}}\right]}\right)+f\left({\left[{ \begin{tabular}{c c}  e & f\\ g & h\end{tabular}}\right]}\right)$.\\
\\Hence, the function $f:M_2(\mathbb{R}) \to \mathbb{R}$ defined by $f\left({\left[{ \begin{tabular}{c c}  a & b\\ c & d\end{tabular}}\right]}\right) = a+b$ is a homomorphism.
\\
\vskip 0.2in

\#66. Suppose $f:G \to K$ and $g:K \to H$ are homomorphisms of groups G, K, and H. Prove that the function $g \circ f$ is a homomorphism from G to H.\\
\\Proof: Let $a,b\in G$. We want to show that $g\circ f(ab)=g\circ f(a)g\circ f(b)$. By composition of functions, $g\circ f(ab)=g(f(ab))$. Since $f$ is a homomorphism, $g(f(ab))=g(f(a)f(b))$. Since $f:G\to K$ and $g:K\to H$, $f(a),f(b)\in K$. Since $g$ is a homomorphism, $g(f(a)f(b))=g(f(a))g(f(b))$. By composition of functions, $g(f(a))=g\circ f(a)$. Similarly, $g(f(b))=g\circ f(b)$. Thus, $g\circ f(ab)=g\circ f(a)g\circ f(b)$. Therefore, $g\circ f$ is a homomorphism from $G\to H$.



\end{document}