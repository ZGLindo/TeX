\documentclass[11pt]{amsart}
\setlength{\parindent}{5mm}
\pagestyle{empty}
\addtolength{\topmargin}{-15mm}
\addtolength{\oddsidemargin}{-20mm}
\addtolength{\evensidemargin}{-20mm}
\setlength{\textheight}{230mm}
\setlength{\textwidth}{175mm}

\begin{document}



 {\bf  Zaki Glenroy Lindo Assignment 3 - Fall 2017 }
 
 \vskip .2in

\noindent Assignment 3: Due Monday September 25, 2017, no later than 8pm.\\

$p. 36-38$ \# 18, 28, 33, 35, 37 (You will be writing three proofs this time so start early and be careful!)\\


\vskip .5in

\#18. Find the 10 elements of the group $\mathbb{Z}_5 \times \mathbb{Z}_2$ and write out the Cayley table. Recall that its operation uses $+_5$ in the first coordinate and $+_2$ in the second coordinate. Identify the inverse of each element.\\


 \vskip .2in
\begin{tabular}{c | c  c  c  c  c  c  c  c  c  c }
	$\mathbb{Z}_{5}x\mathbb{Z}_{2}$ & (0,0) & (0,1) & (1,0) & (1,1) & (2,0) & (2,1) & (3,0) & (3,1) & (4,0) & (4,1)\\
	\hline
	(0,0) & (0,0) & (0,1) & (1,0) & (1,1) & (2,0) & (2,1) & (3,0) & (3,1) & (4,0) & (4,1)\\
	
	(0,1) & (0,1) & (0,0) & (1,1) & (1,0) & (2,1) & (2,0) & (3,1) & (3,0) & (4,1) & (4,0)\\
	
	(1,0) & (1,0) & (1,1) & (2,0) & (2,1) & (3,0) & (3,1) & (4,0) & (4,1) & (0,0) & (0,1)\\
	
	(1,1) & (1,1) & (1,0) & (2,1) & (2,0) & (3,1) & (3,0) & (4,1) & (4,0) & (0,1) & (0,0)\\
	
	(2,0) & (2,0) & (2,1) & (3,0) & (3,1) & (4,0) & (4,1) & (0,0) & (0,1) & (1,0) & (1,1)\\
	
	(2,1) & (2,1) & (2,0) & (3,1) & (3,0) & (4,1) & (4,0) & (0,1) & (0,0) & (1,1) & (1,0)\\

	(3,0) & (3,0) & (3,1) & (4,0) & (4,1) & (0,0) & (0,1) & (1,0) & (1,1) & (2,0) & (2,1)\\
	
	(3,1) & (3,1) & (3,0) & (4,1) & (4,0) & (0,1) & (0,0) & (1,1) & (1,0) & (2,1) & (2,0)\\
	
	(4,0) & (4,0) & (4,1) & (0,0) & (0,1) & (1,0) & (1,1) & (2,0) & (2,1) & (3,0) & (3,1)\\
	
	(4,1) & (4,1) & (4,0) & (0,1) & (0,0) & (1,1) & (1,0) & (2,1) & (2,0) & (3,1) & (3,0)\\
\end{tabular}


\vskip .2in
Inverses:\\
\\
\\
$(0,0)^{-1} = (0,0)\\
(0,1)^{-1} = (0,1)\\
(1,0)^{-1} = (4,0)\\
(1,1)^{-1} = (4,1)\\
(2,0)^{-1} = (3,0)\\
(2,1)^{-1} = (3,1)\\
(3,0)^{-1} = (2,0)\\
(3,1)^{-1} = (2,1)\\
(4,0)^{-1} = (1,0)\\
(4,1)^{-1} = (1,1)$\\
\vskip .2in

\#28. Suppose G is a group and $a,b \in G$. {\bf Prove}: If $a^3 = b$ then $b = aba^{-1}$. \\
\\
{\bf Proof}: Assume $a^3=b$. Then we have $aaa=b$. Now, $aaaa^{-1}=ba^{-1}$. By definition of inverse, we have $aae_{G}=ba^{-1}$. By definition of identity, $aa=ba^{-1}$. We can now say that $aaa=aba^{-1}$. As $aaa=b$, we now have $b=aba^{-1}$. Therefore, if $a^3 = b$ then $b = aba^{-1}$.



 \vskip .2in
 
 \#33. Find the order of each element in the group $A = \left\{ {\left[{ \begin{tabular}{c c}  1 & 0\\ 0 & -1\end{tabular}}\right], \left[{ \begin{tabular}{c c}  1 & 0\\ 0 & 1\end{tabular}}\right], \left[{ \begin{tabular}{c c}  -1 & 0\\ 0 & 1\end{tabular}}\right], \left[{ \begin{tabular}{c c}  -1 & 0\\ 0 & -1\end{tabular}}\right]} \right\}$ under matrix multiplication. Show your work!\\
 (Notice how I created the matrices in TeX to help you, feel free to copy and paste code as needed!)\\
  \vskip .2in
ord($\left[{ \begin{tabular}{c c}  1 & 0\\ 0 & -1\end{tabular}}\right]$):\\
\\
\\
\\
$\left[{ \begin{tabular}{c c}  1 & 0\\ 0 & -1\end{tabular}}\right] X \left[{ \begin{tabular}{c c}  1 & 0\\ 0 & -1\end{tabular}}\right] = \left[{ \begin{tabular}{c c}  1 & 0\\ 0 & 1\end{tabular}}\right]$ \\
\\
\\
\\
Thus, ord($\left[{ \begin{tabular}{c c}  1 & 0\\ 0 & -1\end{tabular}}\right]$) = 2.\\
\\
\\
\\
ord($\left[{ \begin{tabular}{c c}  1 & 0\\ 0 & 1\end{tabular}}\right]$) = 1 as $\left[{ \begin{tabular}{c c}  1 & 0\\ 0 & 1\end{tabular}}\right]$ is the identity matrix.\\
\\
\\
\\
ord($\left[{ \begin{tabular}{c c}  -1 & 0\\ 0 & 1\end{tabular}}\right]$):\\
\\
\\
\\
$\left[{ \begin{tabular}{c c}  -1 & 0\\ 0 & 1\end{tabular}}\right] X \left[{ \begin{tabular}{c c}  -1 & 0\\ 0 & 1\end{tabular}}\right] = \left[{ \begin{tabular}{c c}  1 & 0\\ 0 & 1\end{tabular}}\right]$\\
\\
\\
\\
Thus, ord($\left[{ \begin{tabular}{c c}  -1 & 0\\ 0 & 1\end{tabular}}\right]$) = 2.\\
\\
\\
\\
ord($\left[{ \begin{tabular}{c c}  -1 & 0\\ 0 & -1\end{tabular}}\right]$):\\
\\
\\
\\
$\left[{ \begin{tabular}{c c}  -1 & 0\\ 0 & -1\end{tabular}}\right] X \left[{ \begin{tabular}{c c}  -1 & 0\\ 0 & -1\end{tabular}}\right] = \left[{ \begin{tabular}{c c}  1 & 0\\ 0 & 1\end{tabular}}\right]$\\
\\
\\
\\
Thus, ord($\left[{ \begin{tabular}{c c}  -1 & 0\\ 0 & -1\end{tabular}}\right]$) = 2.\\

\vskip .2in
\#35. Complete the Proof of Theorem 1.26 (p. 28 of the text).\\
\\
\\Suppose G is a group and $a\in G$ with $ord(a)=n$.
{\bf Proof}: ($\leftarrow$) Assume n evenly divides t. That is, $t=nq$ for some integer $q$. We want to show that for any integer $t$, $a^{t}=e_{G}$. As $ord(a)=n$, $a^{n}=e_{G}$. Now, $(a^{n})^{q}=a^{nq}$. As $a^{n}=e_{G}$, $a^{nq}=(e_{G})^{q}$. Now, by definition of identity and our power rules, $(e_{G})^{q}=e_{G}$. Thus, $a^{nq}=e_{G}$. As $t=nq$, $a^{nq}=a^t$. Therefore, $a^t=e_{G}$. Thus, for any integer t, if n evenly divides t, $a^{t}=e_{G}$.\\
\\
With this, we can now conclude that for any integer t, $a^t=e_{G}$ if and only if $n$ evenly divides $t$.

 \vskip .2in



\#37. Suppose G is a group and $a \in G$. Assume $a^{50} = e_G$ but $a^{75} \neq e_G$ and $a^{10} \neq e_G$. Find the order of $a$  and prove that your answer is correct.\\
\\
$ord(a)=50$
\\
\\
{\bf Proof}: Assume $a^{50}=e_{G}$ but $a^{75} \neq e_G$ and $a^{10} \neq e_G$. We want to show that $ord(a)=50$. As we have already assumed that $a^{50} = e_G$, we only need to show that 50 is the least positive integer n such that $a^n=e_G$. By Theorem 1.26, we have to show that for each factor of 50, $k$, $a^k\neq e_G$.\\
\\
The factors of 50 are: 1, 2, 5, 10, and 25.\\
\\
Case 1: k = 1. \\
If $a^1=e_G$, then $aaaaaaaaaa=a^{10}=e_G$. As $a^{10}\neq e_G$, $a^1\neq e_G$.\\
\\
Case 2: k = 2.\\
If $a^2=e_G$, then $a^2a^2a^2a^2a^2=a^{10}=e_G$. As $a^{10}\neq e_G$, $a^2\neq e_G$.\\
\\
Case 3: k = 5.\\
If $a^5=e_G$, then $a^5a^5=a^{10}=e_G$. As $a^{10}\neq e_G$, $a^5\neq e_G$.\\
\\
Case 4: k = 10.\\
As $a^{10}\neq e_G$, $a^10\neq e_G$.\\
\\
Case 5: k = 25.\\
If $a^25=e_G$, then $a^25a^25a^25=a^{75}=e_G$. As $a^{75}\neq e_G$, $a^25\neq e_G$.\\
\\
Thus, as we have showed that for each factor of 50, $k$, $a^k\neq e_G$, we know that 50 is the least positive integer n such that $a^n=e_G$. Thus, $ord(a)=50$.
\end{document}